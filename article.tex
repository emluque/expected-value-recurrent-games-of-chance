\documentclass[12pt,reqno]{amsart}
\usepackage{fullpage}
\usepackage{amsfonts}
\usepackage{amssymb}
\usepackage{times}
\usepackage{graphicx}
\vfuzz=2pt

% some "funny lines" referred to later:
\newtheorem{thm}{Theorem}[section]
\newtheorem{cor}[thm]{Corollary}
\newtheorem{lem}[thm]{Lemma}
\newtheorem{prop}[thm]{Proposition}
{ \theoremstyle{remark}\newtheorem*{remark}{Remark} }

\newcommand{\class}{MATH 613E: Topics in Analytic Number Theory}
\newcommand{\population}[2]{There are #2 students giving lectures, and also #2 students writing up notes from lectures. Everybody in the class is very #1 about this.}
\newcommand{\tex}{{\tt .tex}}

\begin{document}

\title[On the expected value of recurrent games of chance]{On the expected value of recurrent games of chance}
% or if you want, simply \title{Title of the article}
\author{Emiliano Martínez Luque}
\email{martinezluque@gmail.com}
\maketitle

\begin{abstract}
We define a recurrent game of chance, then provide formulas for calculating it's expected value
\end{abstract}

\section{Introduction}
\label{first section}

Recurrent game of chance definition.

A betting game that starting with an original bet and a set of probabilities associated with differente payoffs recursively applies this game conditions to the result of each succesive step of the game.

x\textsubscript{0} initial ammount

\section{simple example}

x\textsubscript{0} is the original ammount.

Two possible outcames for each step with equal probability defined as having the current ammount multiplied by one of two coefficients (b, b') with equal probability.

p(b) = p(b') = 0.5

b \textgreater b' \textgreater 0



Result of the recurrence at infinity:

if b' \textgreater 1/b tends to infinity

if b' \textless 1/b tends to 0

if b'= 1/b tends to x\textsubscript{0}



Explain result geometrically--

Rant about b' = 1/b is the exact definition of a fair game.. go on about pascal triangle and binomial distribution

\section{multiple outcames with equal probability}

m number of possible outcames, each with same probability and associated with a coefficient b\textsubscript{1} to b\textsubscript{m}



Result of the recurrence at infinity:



Let's define s = $\prod_{i=1}^{m} b_{i}$

if s \textgreater 1 tends to infinity

if s \textless 1 tends to 0

if s = 1 tends to x\textsubscript{0}


Prove algebraically.

\section{multiple outcames each with it's own probability}

m number of possible outcames, each with it's own coefficient b\textsubscript{i} and it's own probability p\textsubscript{i}

Result of the recurrence at infinity:

Let's define s = $\prod_{i=1}^{m} mp_{i}b_{i}$



if s \textgreater 1 tends to infinity

if s \textless 1 tends to 0

if s = 1 tends to x\textsubscript{0}



Examine distributions of probabilities.. this one is harsher.. to prove



Define expected value of recurrence r\textsubscript{n}

r\textsubscript{0} = x\textsubscript{0}

r\textsubscript{n} = sr\textsubscript{n-1}

Explain based on the definition of the common expected value formula


\end{document}
